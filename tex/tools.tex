%Tools and Software Used. 
For this study several externally developed or commercially purchased tools, hardware, and software were used. Each major tool or piece of hardware will be briefly introduced and given a brief description of its source, purpose and method of procurement.

\section{Computing Hardware}
\todo{Cite the spec sheet for These parts!}The analyses presented in this report were obtained using commercially available computing hardware. The computer's relevant specifications are given below: 

\begin{itemize}
\item CPU: AMD Ryzen 2400G. 4 Processing Cores with 8 execution threads. Frequency during tests: 3.8GHz
\item Memory: 16GB DDR4 Memory at a frequency of 2666 MHz
\item Storage: Intel 540 Series SSD. Capacity:240GB, up to 540 MBps read speed, 490 MBps write
\end{itemize}

While this computer is a general purpose unit and sees daily use outside the scope of this report, no other tasks were performed simultaneously with the workloads presented herein. Execution times presented represent near-maximum performance for these workloads. 

\section{Software}
\subsection{NASTRAN}
The name NASTRAN is short for \textbf{NA}SA \textbf{St}ructural \textbf{An}alysis. It is a finite element solver originally developed by NASA. It enjoys wide popularity throughout industry and sees use for performing linear and nonlinear structural analyses on a variety of materials. 

The version of NASTRAN employed in the code presented herein is a recently open-sourced copy of the NASTRAN95 solver. As the name implies, it is a version originally developed in 1995. While newer versions of this software exist, this is the newest copy that is freely available. For simple solutions such as those needed for this work, NASTRAN95 is functionally very similar to the more modern commercial utilities available. This version of NASTRAN is freely available on GitHub. \todo{Whole section needs citations.} 

\subsection{msslhs}

msslhs is a Python utility that provides latin hypercube sampling in several locations throughout the code presented. This utility is created and maintained by Justin Hughes. It is available from the Mississippi State University CyberDesign Wiki \cite{msslhs}. 
