The purpose of this study was to implement the two candidate methods and evaluate the quality of the resultant Pareto Fronts. With the results contained in the previous section, it is possible to make this comparison. Additionally, possible avenues for future work will be presented.  

\section{Algorithm Comparison}
Based on the results obtained in the previous section, it is clear that both algorithms function and provide similar Pareto Fronts for the parameters requested. Both algorithms developed are therefore apparently suitable for performing the type of analysis this study is interested in.

While the Pareto Fronts are of a similar quality, the per-generation run times for the two are very different. Due to the smaller number of analyses per generation for the Stochastic loads algorithm, this algorithm had much shorter per-generation times than the Aggregated LHS. The example system in this study turned out to converge quickly for both algorithms. However, for more complex problems that converge slowly, the per-generation computation time may be an advantage for the Stochastic loads algorithm. 

On the other hand, the Aggregated LHS algorithm had wider and more complete coverage of the Pareto Front than the Stochastic Loads algorithm. This may be an advantage for applications where a best-fit curve for the Pareto Front is required, or in any case where dense coverage of the Pareto Front is preferred. 

\section{Future Work}
Since both algorithms were able to successfully and repeatably identify the Pareto Frontier for the Example Problem provided, additional work is justifiable to continue to study and refine the implementation of both algorithms. Specifically, the accuracy of the solution obtained could be improved by utilizing geometry optimization instead of parameter-based optimization to control the configuration of the beam. This would allow additional designs to be investigated and may generate more effective designs.

It would also be worth while to investigate the cause of the denser coverage of the Pareto Frontier by the Aggregated LHS algorithm. this may allow this denser coverage to be used in Stochastic Loads as well. 

The investigation of the use of Multiobjective Stopping conditions to stop execution early would also be a possible next step. this would allow for early termination of the evolution if convergence is reached. 

Finally, It may be worthwhile to investigate how compliance with industry standards can be used as a constraint on this type of algorithm. This would further reduce workload on the designer by improving the chances that all points on the Pareto Front are safe and legal to produce and use. 
