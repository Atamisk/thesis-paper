With the results contained in the previous section, it is now possible to draw some conclusions about the efficacy and efficiency of the two algorithms. In this chapter, the two methods will be compared and discussed. Additionally, possible avenues for future work will be presented.  

\section{Algorithm Comparison}
Based on the results obtained in the previous section, it is clear that both algorithms function and provide a Pareto Frontier for the parameters requested. Both algorithms developed are therefore apparently suitable for performing the type of analysis this study is interested in.

In Figure \ref{fig:pfront_comp_long}, the two long runs of both methods are shown superimposed on each other. It can be seen in this figure that given sufficient solution time, both algorithms arrive at the same approximate Pareto Frontier curve. However, this figure also shows the Aggregate LHS solution having generally a wider coverage of the Frontier's curve, particularly in masses below 300 kg. 

Additionally, Figure \ref{fig:pfront_comp_short} shows a comparison of the Pareto Frontiers of two short runs. Here, it is clear that the Stochastic Loads method has a loss in accuracy in the shorter run. It appears that the relatively small number of generations and individuals that the Stochastic Loads run was required to employ in order to meet the time restrictions have contributed to this loss in accuracy. It is also clear from the Solution Statistics tables in the previous section that the per-generation solution time for Stochastic loads is nearly twice that of the Aggregated LHS Method. The primary source of this discontinuity in per-generation times is the additional calculations that are required to calculate the Reliability Index. Because the two unit response tensors discussed in Section \ref{sec:beta} require NASTRAN to run twice, each generation has twice as many calls to NASTRAN as a generation of Aggregate LHS. 

\section{Future Work}
Since both algorithms were able to successfully and repeatably identify the Pareto Frontier for the Example Problem provided, additional work is justifiable to continue to study and refine the implementation of both algorithms. Specifically, the accuracy of the solution obtained could be obtained by utilizing geometry optimization instead of parameter-based optimization to control the configuration of the beam. This would allow additional designs to be investigated and may generate more effective designs. The implementation of the Reliability Index calculation is, as mentioned earlier, slow when compared to the Aggregated LHS method. The efficiency of the Stochastic Loads algorithm can be improved drastically by improving the quality of the implementation of this portion of the code. 
