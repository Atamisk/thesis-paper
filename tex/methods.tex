With all of the principals having been covered previously, the actual solution methodology can be discussed. This chapter provides an overview of the actual solution process used in the code that was developed. While it does not provide a full analysis and discussion of the entire codebase, it should provide sufficient information for the reader to understand the actual code provided. 

\section{Modeling the base beam}
For both solution methods, a common basic design of the beam to be studied was constructed. For this study, a basic design using two ``C"-style sections to form the two main moment-bearing sections was selected. This style was selected because the cross section lends itself to parameter-based optimization, whereas more complex designs such as box beam spreaders are more well suited to full geometry optimization. Section \ref{sec:beam_des} has details on the design. 

This basic design was modeled in Siemens's Finite Element Pre-processor, FEMAP. The modeled area takes advantage of the two-plane symmetry in the beam to only model one quarter of the beam. It is worth noting that the loading on the beam is affected by the symmetry, and the loads presented for analysis are one fourth of the actual beam loadings. 

\section{Solution Strategies}
Two solution strategies were attempted for the example problem proposed. One was termed as the \emph{Aggregated Latin Hypercube Sampling Approach}, while the other was named the \emph{Stochastic Loads Approach}. The following sections describe each solution method in detail and outline similarities and differences between them. 

\subsection{Aggregated Latin Hypercube Approach}

In a nutshell, the Aggregated Latin Hypercube (ALHS) Approach analyzes a variety of discrete load cases and collects Pareto Front data for each load case. It does this by:

\begin{enumerate}
\item Identifying the most conservative Load Cases to consider
\item Performing MODE Optimization for each identified load cases. This generates a unique Pareto Front for each load case. 
\item Aggregating the Pareto Fronts from all load cases to find the overall best designs across all load cases. 
\end{enumerate}
Each individual step of this process is outlined in detail below. 

\subsubsection{Identify Load Cases}
In order to find load cases for use, a Latin Hypercube Sampling algorithm is used to select 1000 independent loading conditions within the set of possible load conditions, which is assumed to be a set of normally distributed sample spaces. In this case, the most conservative loadings are those significantly above the mean for the magnitude of the applied force. In order to ensure reliability, the set of load cases to be analyzed is restricted to those load cases more than 1.96 standard deviations above the mean value of the applied load. This ensures that all of the load cases considered are in the top 2.5\% of the sample space. This will result in a set of approximately 25 load cases to consider. 
\subsubsection{Analyze load Cases} 
For each load case selected, a set of randomly selected candidate designs are generated. These designs are analyzed using Finite Element Analysis to find the maximum stress and design weight. These values are used as fitness functions to rank the designs. This process is repeated using a Multi Objective Differential Evolution algorithm to selectively refine the designs for a set number of generations.
\subsubsection{Aggregation}
At this point, the solution algorithm has 25 separate load cases with individual Pareto fronts. The designs that make up each Pareto front are added to a single unified set of designs. From there, they are once again ranked according to dominance in accordance with Equation \ref{eq:dominance} and a ``final" Pareto front is generated. The designs that comprise this ``final" Pareto front are considered to be the most desirable designs and are presented to the designer.

\subsection{Stochastic Loads Approach}
The stochastic loads approach is functionally similar to the Aggregated Latin Hypercube Sampling Approach, with a few key differences: 

\begin{enumerate}
\item Only a single load case is considered. 
\item Stress is not directly considered a fitness variable. Instead, the Reliability Index is considered. While the Reliability Index is proportional to stress, it is not a direct measurement. See Section \ref{sec:beta} for details. 
\end{enumerate}
The overall process is detailed below. 

\subsubsection{Load Case}
Instead of using multiple load cases, this approach uses a single load case. However, it is specified as a stochastic set of loads instead of discrete loads applied. For example, the load case considered for the Example System is shown in Table \ref{tab:stoch_params}. Note that this information matches the design parameters given in section \ref{sec:perf_req}. Specifying the load case this way effectively covers the entire performance envelope and makes the single load case valid for all states of load on the beam. 

\begin{table}[!h]
    \caption{Stochastic Parameters for Example System}
    \label{tab:stoch_params}
    \begin{center}
    \begin{tabular}{|l|cc|}
	    \hline
	    Parameter & Mean Value & Standard Deviation\\
	    \hline
	    Hoist Load X Direction & 0N & 13 kN\\
	    Hoist Load Y Direction & 150 kN & 19.5 kN\\
	    \hline
    \end{tabular}
    \end{center}
\end{table}

\subsubsection{Analyze Load Case}
The actual analysis of the load case selected is nearly identical to the ALHS Approach. The key difference is the use of the unitless parameter $\beta$ to represent the stress on the beam. This allows the stress, and by inference the safety factor of the device, to be represented in stochastic terms. As mentioned in section \ref{sec:beta}, the calculation for $\beta$ requires the two unit response tensors $\sigma_{Px}$ and $\sigma_{Py}$ to be retrieved. Currently, this must be done by running NASTRAN twice: Once with a unit load in the x direction, and again with a unit load in the y direction. For a complete discussion and derivation of the parameter $\beta$, see section \ref{sec:beta}. Using $\beta$ and the component weight as fitness measurements, MODE is performed on the load case.

\subsubsection{Reporting}
MODE optimization on the load case defined above generates a single Pareto Front that is reported to the user as recommended designs. Also reported is the Pareto Front presented graphically. 
