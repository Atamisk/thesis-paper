\section{Problem Description}
This particular study seeks to formulate a method to design an equalizer beam using multi-objective optimization. It further seeks to evaluate the performance of the formulated method based on a design based upon standards currently common in the industry.
\todo{Comparing against a traditionally developed beam is a new idea. Is this something that seems appropriate?}
To this end, an example system will be used as a subject for the design. The basic requirement parameters will remain the same throughout both solution methods to ensure a consistent comparison between the 2 methods. The basic parameters for the solution are presented below. 

\subsection{Performance requirements}
In this case, performance requirements primarily relate to the lifting capacity and the allowable side pull on the beam. The requirements selected for this problem are: 
\begin{enumerate}
\item Lifting capacity: 15 metric tons, 15000kg, 33000 lbm, 147 kN. For this problem, 150 kN was used. 
\item Minimum capable side load: $\pm 5^{\circ} $
\end{enumerate}
\subsection{Validation Load}
The selected validation load was chosen based on the targeted lifting capacity of the beam, which is 147 kN. This is the validation load that was chosen. 
\subsection{Validation Criteria}
The validation stress was chosen such that the overall system safety factor is in accordance with ASME BTH-1, for a Design Category B lifting beam. This dictates a factor of safety for the beam to be no more than 33\% of the minimum yield stress of the material. In this case, we are assuming the use of ASTM A-572 Grade 42 Steel with a yield point of 289 MPa. 
In this case, this equates to:
\begin{align*}
	\sigma_{\mathit{allow}} &= 0.33 \cdot S_{y}\\
	\sigma_{allow} &= 0.33 \cdot 289  \text{ MPa} \\
	\sigma_{allow} &= 95.3 \text{ MPa}
\end{align*}
