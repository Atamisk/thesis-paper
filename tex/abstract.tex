{\LARGE\center
  \textbf{\uppercase{Abstract}}\par
}

{\Large{ {\centerline{Differentially Evolved design of an Equalizer Beam}}}}
{\Large{ {\centerline{for use in General Purpose Lifting and Handling}}}}

\vspace{1cm}

This study investigated two methods for using Multiple Objective Differential Evolution (MODE) and Finite Element Analysis (FEA) together to algortihmically develop designs for crane equalizer beams. The Finite Element data was used to drive fitness functions for a MODE analysis by 2 similar but seperate methods. The methods were investigated for execution speed and accuracy to one another. 

The study showed that both algorithm arrived at roughly the same solution in all cases presented to the solver, with the exception of very short runs that did not achieve optimal solutions. One solver was clearly more time efficient, but this is more likely the result of implementation details than a fundamental failing in the slower algorithm. 

The methods presented herein could potentially see application in the design of crane lifting hardware, as well as most structural objects subject to analysis and design based on static response solvable with a NASTRAN Linear Static solution. 
