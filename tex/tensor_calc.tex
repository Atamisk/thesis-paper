\section{Calculating $\beta$}
The tensor definition of the von mises stress can be written as:
   \begin{equation}
      \sigma' = \sqrt{\frac{3}{2} \cdot \left(\sigma_{dev}:\sigma_{dev} \right)}
   \end{equation}
Where $\sigma_{dev}$ is the stress deviator tensor, given in terms of the overall stress tensor $\sigma$ as: 
   \begin{equation}
      \sigma_{dev} = \sigma - \frac{1}{3} \mathrm{tr}(\sigma) \left[ \mathbf{I} \right]
   \end{equation}
For purposes that will become clear later, we can call this operation a matrix operator $\gamma$:
   \begin{align}
      \gamma(x) &= x - \frac{1}{3} \mathrm{tr}(x) \left[ \mathbf{I} \right]\\
      \sigma_{dev} &= \gamma(\sigma) \label{eq:gamma}
   \end{align}
In this study, $\sigma$ is constructed from the component response tensors, $\sigma_{Px}$ and $\sigma_{Py}$. Sigma is then generated by: 
   \begin{equation}
      \sigma = \sigma_{Px} \cdot P_x + \sigma_{Py} \cdot P_y
   \end{equation}
Applying equation \ref{eq:gamma} to the above definition of $\sigma$ yields:
   \begin{align}
	   \gamma_(\sigma) &= \left(\sigma_{Px} \cdot P_x + \sigma_{Py} \cdot P_y\right) - 
                       \frac{1}{3} \mathrm{tr} \left(\sigma_{Px} \cdot P_x + \sigma_{Py}
                       \cdot P_y\right) \left[ \mathbf{I} \right] 
   \end{align}
From here, it is important to remember that taking the trace of a matrix is a distributive operation, as is multiplying a scalar and a matrix. Therefore, the above equation can be rewritten as: 
   \begin{align}
	   \gamma(\sigma) &= \left(\sigma_{Px} \cdot P_x + \sigma_{Py} \cdot P_y\right) - 
                       \frac{1}{3} \left(\mathrm{tr} \left(\sigma_{Px} \cdot P_x\right) +
                       \mathrm{tr} \left( \sigma_{Py} \cdot P_y\right)\right) 
                       \left[ \mathbf{I} \right]\nonumber \\
                      &= \left(\sigma_{Px} \cdot P_x + \sigma_{Py} \cdot P_y\right) - 
		       \frac{1}{3} \left(\mathrm{tr} \left(\sigma_{Px} \cdot P_x\right)
		       \left[ \mathbf{I} \right] +
                       \mathrm{tr} \left( \sigma_{Py} \cdot P_y\right) 
                       \left[ \mathbf{I} \right]\right)\nonumber\\
		      &= \left(\sigma_{Px} \cdot P_x + \sigma_{Py} \cdot P_y\right) - 
		       \frac{1}{3} \left(P_x \cdot \mathrm{tr} \left(\sigma_{Px}\right)
		       \left[ \mathbf{I} \right] +
                       P_y \cdot \mathrm{tr} \left( \sigma_{Py} \right) 
                       \left[ \mathbf{I} \right]\right)\nonumber\\
		      &= P_x \cdot \left( \sigma_{Px} - \frac{1}{3} \mathrm{tr}(\sigma_{Px})
			 \left[ \mathbf{I} \right] \right) + P_y \cdot \left( \sigma_{Py} -
			 \frac{1}{3} \mathrm{tr}(\sigma_{Py})
			 \left[ \mathbf{I} \right] \right)\nonumber\\
		      &= P_x \cdot \gamma(\sigma_{Px}) + P_y \cdot \gamma(\sigma_{Py})
   \end{align}
Note that the terms $\gamma(\sigma_{Px})$ and $\gamma(\sigma_{Py})$ do not contain any terms related to $P_x$ or $P_y$. This implies that they are constant when deriving with respect to these variables. From this point forward, we will refer to these deviatoric unit tensors as:

   \begin{align*}
	   \gamma(\sigma_{Px}) &= \sigma_{devx}\\
	   \gamma(\sigma_{Py}) &= \sigma_{devy}
   \end{align*}
